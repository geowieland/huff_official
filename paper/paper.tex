% Options for packages loaded elsewhere
\PassOptionsToPackage{unicode}{hyperref}
\PassOptionsToPackage{hyphens}{url}
\documentclass[
]{article}
\usepackage{xcolor}
\usepackage{amsmath,amssymb}
\setcounter{secnumdepth}{-\maxdimen} % remove section numbering
\usepackage{iftex}
\ifPDFTeX
  \usepackage[T1]{fontenc}
  \usepackage[utf8]{inputenc}
  \usepackage{textcomp} % provide euro and other symbols
\else % if luatex or xetex
  \usepackage{unicode-math} % this also loads fontspec
  \defaultfontfeatures{Scale=MatchLowercase}
  \defaultfontfeatures[\rmfamily]{Ligatures=TeX,Scale=1}
\fi
\usepackage{lmodern}
\ifPDFTeX\else
  % xetex/luatex font selection
\fi
% Use upquote if available, for straight quotes in verbatim environments
\IfFileExists{upquote.sty}{\usepackage{upquote}}{}
\IfFileExists{microtype.sty}{% use microtype if available
  \usepackage[]{microtype}
  \UseMicrotypeSet[protrusion]{basicmath} % disable protrusion for tt fonts
}{}
\makeatletter
\@ifundefined{KOMAClassName}{% if non-KOMA class
  \IfFileExists{parskip.sty}{%
    \usepackage{parskip}
  }{% else
    \setlength{\parindent}{0pt}
    \setlength{\parskip}{6pt plus 2pt minus 1pt}}
}{% if KOMA class
  \KOMAoptions{parskip=half}}
\makeatother
% definitions for citeproc citations
\NewDocumentCommand\citeproctext{}{}
\NewDocumentCommand\citeproc{mm}{%
  \begingroup\def\citeproctext{#2}\cite{#1}\endgroup}
\makeatletter
 % allow citations to break across lines
 \let\@cite@ofmt\@firstofone
 % avoid brackets around text for \cite:
 \def\@biblabel#1{}
 \def\@cite#1#2{{#1\if@tempswa , #2\fi}}
\makeatother
\newlength{\cslhangindent}
\setlength{\cslhangindent}{1.5em}
\newlength{\csllabelwidth}
\setlength{\csllabelwidth}{3em}
\newenvironment{CSLReferences}[2] % #1 hanging-indent, #2 entry-spacing
 {\begin{list}{}{%
  \setlength{\itemindent}{0pt}
  \setlength{\leftmargin}{0pt}
  \setlength{\parsep}{0pt}
  % turn on hanging indent if param 1 is 1
  \ifodd #1
   \setlength{\leftmargin}{\cslhangindent}
   \setlength{\itemindent}{-1\cslhangindent}
  \fi
  % set entry spacing
  \setlength{\itemsep}{#2\baselineskip}}}
 {\end{list}}
\usepackage{calc}
\newcommand{\CSLBlock}[1]{\hfill\break\parbox[t]{\linewidth}{\strut\ignorespaces#1\strut}}
\newcommand{\CSLLeftMargin}[1]{\parbox[t]{\csllabelwidth}{\strut#1\strut}}
\newcommand{\CSLRightInline}[1]{\parbox[t]{\linewidth - \csllabelwidth}{\strut#1\strut}}
\newcommand{\CSLIndent}[1]{\hspace{\cslhangindent}#1}
\setlength{\emergencystretch}{3em} % prevent overfull lines
\providecommand{\tightlist}{%
  \setlength{\itemsep}{0pt}\setlength{\parskip}{0pt}}
\usepackage{bookmark}
\IfFileExists{xurl.sty}{\usepackage{xurl}}{} % add URL line breaks if available
\urlstyle{same}
\hypersetup{
  pdftitle={huff: A Python package for Market Area Analysis},
  pdfauthor={ Freiburg, Germany ORCID: 0000-0001-5168-9846 EMail: geowieland@googlemail.com},
  hidelinks,
  pdfcreator={LaTeX via pandoc}}

\title{huff: A Python package for Market Area Analysis}
\author{\textbf{Thomas Wieland}\\
Freiburg, Germany\\
ORCID: 0000-0001-5168-9846\\
EMail: geowieland@googlemail.com}
\date{19 February 2026}

\begin{document}
\maketitle

\textbf{Summary}

Market area models, such as the \emph{Huff model} and its extensions,
are widely used to estimate regional market shares and customer flows of
retail and service locations. Another, now very common, area of
\hspace{0pt}\hspace{0pt}application is the analysis of catchment areas,
supply structures and the accessibility of healthcare locations. The
\texttt{huff} Python package provides a complete workflow for market
area analysis, including data import, construction of origin-destination
interaction matrices, basic model analysis, parameter estimation from
empirical data, calculation of distance or travel time indicators, and
map visualization. Additionally, the package provides several methods of
spatial accessibility analysis. The package is modular and
object-oriented. It is intended for researchers in economic geography,
regional economics, spatial planning, marketing, geoinformation science,
and health geography. The software is openly available via the
\href{https://pypi.org/project/huff/}{Python Package Index (PyPI)}; its
development and version history are managed in a public
\href{https://github.com/geowieland/huff_official}{GitHub Repository}
and archived at \href{https://doi.org/10.5281/zenodo.18639559}{Zenodo}.

\textbf{Statement of need}

Market area models are used in economic geography, regional economics,
spatial planning, geoinformation science, and marketing, enabling the
analysis and forecasting of market areas and customer flows for retail
and service locations. The classical and most popular approach is the
\emph{Huff model} (Huff 1962, 1963, 1961) and its numerous derivates and
extensions, such as the \emph{Multiplicative Competitive Interaction
(MCI) Model} (Nakanishi and Cooper 1974, 1982). Typical research
applications include examining the influence of store attributes and
transport costs on consumer store choice, forecasting the revenue of new
locations, or predicting the impact of new locations on existing ones
(De Beule et al. 2014; Fittkau 2004; Li and Liu 2012; Mensing 2018; Oruc
and Tihi 2012; Suárez-Vega et al. 2015; Wieland 2015, 2019).

In health geography, such models are used to analyse catchment areas
with respect to medical practices and hospitals (Bai et al. 2023; Fülop
et al. 2011; Jia 2016; Latruwe et al. 2023; Rhein et al. 2025; Wieland
2018), and they are also increasingly being linked to methods for
analyzing the supply structure and accessibility of health locations
(Liu L 2022; Rauch et al. 2023; Subal J 2021). Moreover, market area
models are also applied to other location-related contexts such as
airports or recreation facilities (Wang et al. 2022; Wang et al. 2026).

There are several major challenges in model-based market area analyses:

\begin{itemize}
\item
  The calibration of the Huff model based on observed data on consumer
  behavior and/or store sales is difficult because the model is
  nonlinear in its weighting parameters (Huff 2003; Wieland 2017). In
  this context, the MCI model (Nakanishi and Cooper 1974, 1982) has been
  developed as an econometric estimation technique based on a
  linearization (\emph{log-centering transformation}). As this approach
  requires empirical market shares for fitting, it is applied in cases
  where customer-store interaction data was obtained by surveys or
  secondary data (Baviera-Puig et al. 2016; Latruwe et al. 2023; Oruc
  and Tihi 2012; Suárez-Vega et al. 2015; Wieland 2015, 2019). Several
  other researchers developed and used nonlinear iterative fitting
  approaches, especially when no empirical customer-store interactions
  are available, but only total sales of the locations investigated (De
  Beule et al. 2014; Güßefeldt 2002; GH et al. 1972; Li and Liu 2012;
  Liang et al. 2020; Mensing 2018; Orpana and Lampinen 2003; Wieland
  2017). Due to the pronounced sensitivity of market area models to
  weighting schemes, the availability of multiple calibration approaches
  is essential in market area analysis.
\item
  Researchers must choose and compare appropriate weighting functions,
  which may be chosen based on theoretical considerations and may result
  in substantially different results. Nowadays, for input variables such
  as travel time, several weighting functions (e.g., power, exponential,
  logistic) are used, and the model results are compared using
  goodness-to-fit metrics (Bai et al. 2023; Latruwe et al. 2023; Li and
  Liu 2012; Orpana and Lampinen 2003). It is, thus, necessary that,
  within the market area analysis workflow, several weighting functions
  are available, and that there are options to compare different model
  specifications based on fit metrics.
\item
  Calculating travel times may be time consuming because these are based
  on graph theory network analysis and require real street networks
  (Huff and McCallum 2008). Therefore, market area analysis typically
  requires GIS (Geographic Information System) support and/or access to
  an API providing calculations based on input origins and destinations.
  It is extremely helpful for researchers if they can also complete this
  part of the market area analysis workflow within the analysis tool.
\end{itemize}

The huff package for Python v1.8.x essentially provides the following
features:

\begin{itemize}
\item
  \emph{Data management and preliminary analysis}: Users may load
  customer origins and supply locations from point shapefiles (or CSV,
  XLSX). Attributes of customer origins and supply locations (variables,
  weightings) may be set by the user. The next step is to create an
  \emph{interaction matrix} with a built-in function, on the basis of
  which all implemented models can then be calculated. Within an
  interaction matrix, \emph{transport costs} (distance or travel time
  between customer origins and supply locations) may be calculated with
  built-in methods.
\item
  \emph{Basic Huff model analysis}: Given an interaction matrix, users
  may calculate probabilities and expected customer flows with respect
  to customer origins, and total market areas of supply locations.
\item
  \emph{Parameter estimation based on empirical data}: Given empirical
  data on customer flows, regional market shares, or total sales, users
  may estimate weighting parameters of market area models. Model
  parametrization may be undergone using the econometric approach in the
  \emph{MCI model} (if regional market shares are available) or by
  Maximum Likelihood optimization using regional market shares, customer
  flows, or total market areas.
\item
  \emph{Accessibility analysis}: The package includes methods of
  accessibility analysis, which may be combined with market area
  analysis (especially empirical estimation of weighting parameters),
  namely \emph{Hansen accessibility} (Hansen 1959) and \emph{Two-step
  floating catchment area analysis (2SFCA)} (Luo and Wang 2003).
\item
  \emph{GIS tools}: The library also includes auxiliary GIS functions
  for market area analysis (buffer, distance matrix, overlay statistics)
  and clients for OpenRouteService (Neis and Zipf 2008) and
  OpenStreetMap (Haklay 2008) for simple maps, with all of them being
  implemented in the market area analysis functions but are also able to
  be used stand-alone.
\end{itemize}

\textbf{State of the field}

To the best of our knowledge, no open-source Python package currently
provides market area analysis and parameter estimation for the Huff or
MCI model. No open-source software package currently exists that covers
the entire workflow of market area analyses, as described in the
``Statement of need'' section. Some but not all of the functionalities
mentioned are implemented in R packages: Both the
\texttt{SpatialPosition} package (Giraud and Commenges 2025) and the
\texttt{huff-tools} package (Pavlis et al. 2014) provide basic Huff
Model analyses with two parameters, calculation of air distances, and
map visualization. The R package \texttt{MCI} (Wieland 2017) focuses on
model fitting based on empirical data, but does not provide processing
of geospatial data and the calculation of distances or travel times.
Accessibility analysis via two-step floating catchment area analysis is
implemented in the R package \texttt{accessibility} (Pereira and
Herszenhut 2024). The (almost) complete workflow for market area
analyses using the Huff/MCI model is currently only implemented in
proprietary GIS software, namely the \emph{ArcGIS Business Analyst} by
\emph{ESRI} (Esri 2025; Huff and McCallum 2008).

\textbf{Software Architecture}

The \texttt{huff} package is organized into a modular architecture that
separates core modeling functionality from auxiliary helper modules. All
model-related classes, methods and functions are implemented in the
\texttt{models} module. Supporting functionalities are provided in
separate modules, organized thematically. For example, the \texttt{ors}
module provides an OpenRouteService client for retrieving travel time
matrices and isochrones, which may be directly accessed from the
\texttt{models} module. This design allows auxiliary functions to be
used independently of the core models (stand-alone). In order to
harmonize the data and outputs while processing, the \texttt{config}
module includes configurations for all functions and definitions of
default column names, suffixes and prefixes, and model terminology.

The \texttt{huff} library follows an object-oriented design. The class
structure reflects the conceptual actors of a spatial market: Customer
demand locations are represented by the \texttt{CustomerOrigins} class
and supply locations by the \texttt{SupplyLocations} class. Their
connection is established via an interaction matrix containing all
possible origin-destination combinations and the corresponding data,
such as travel times and location attributes. It is created from the
location data using the built-in function
\texttt{create\_interaction\_matrix()} from the \texttt{models} module,
resulting in an instance of the \texttt{InteractionMatrix} class. All
implemented model analyses may be calculated from an
\texttt{InteractionMatrix} object, with the individual steps of the
model calculations being methods of this class, e.g.,
\texttt{transport\_costs()} for adding distances or travel times,
\texttt{probabilities()}, \texttt{flows()}, and \texttt{marketareas()}
for Huff model calculations, or \texttt{mci\_fit()} for a MCI model
analysis. These model analyses return objects of specific classes for
each model, e.g., \texttt{HuffModel} and \texttt{MCIModel} for Huff and
MCI models, respectively. All mentioned classes include
\texttt{summary()}, \texttt{show\_log()}, and, in relevant cases,
\texttt{plot()} methods.

This structure was chosen to ensure a consistent workflow and a unified
data structure, regardless of which model analysis is to be performed.
The typical workflow for a basic Huff analysis (without empirical
parameter estimation) consists of the following steps: (1) Load
geospatial data of customer origins and supply locations, (2) Define
their attributes and weightings, (3) Create an interaction matrix from
origins and destinations, including the calculation of distances or
travel time, (4) Calculate regional market shares, expected customer
flows, and total market areas of all supply locations (This workflow is
shown in the \emph{Examples} section of the package README.MD). Advanced
model analyses (e.g., including empirical calibration) require further
steps (See the examples folder in the
\href{http://www.github.com/geowieland/huff_official.git}{corresponding
GitHub repository}).

\textbf{Research impact statement}

The \texttt{huff} package is currently used in a health geography
project at the Wuerzburg university hospital which deals with the
catchment areas of pediatric oncology care; a paper on this topic is
currently in the review process. Given the rising number of scientific
studies using market area models - particularly for non-retail purposes
such as health geography - and the widespread use of Python as a
programming language, it is to be expected that the \texttt{huff}
library will see frequent adoption in related research projects.

\textbf{Software development history statement}

Due to data confidentiality requirements, the early development of the
\texttt{huff} library took place in a private repository; the public
repository was initialized more recently to provide open access for
reproducibility and review. The \texttt{huff} Python package has been
publicly developed and published via the {[}Python Package Index{]}
(https://pypi.org/project/huff/) since April 2025. As of submission, it
has undergone 41 releases, showing continuous improvement and feature
additions. The library is actively used: since its first release
(version 1.0.0) in April 2025, it has been downloaded approximately
21,900 times from the Python Package Index (source:
\href{https://pepy.tech/project/huff}{pepy.tech}, accessed February 14,
2026).

\textbf{AI usage disclosure}

No AI tools were used for software design, implementation, or
decision-making. The Continue agent in Microsoft Visual Studio Code
(with model GPT-5 mini) was used to generate initial docstrings, which
were subsequently reviewed and adapted by the author. The manuscript
text was written without the use of AI tools.

\textbf{References}

\protect\phantomsection\label{refs}
\begin{CSLReferences}{1}{1}
\bibitem[\citeproctext]{ref-bai2023}
Bai, Lingyao, Zhuolin Tao, Yang Cheng, Ling Feng, and Shaoshuai Wang.
2023. {``{Delineating hierarchical obstetric hospital service areas
using the Huff model based on medical records}.''} \emph{Applied
Geography} 153: 102903.
https://doi.org/\url{https://doi.org/10.1016/j.apgeog.2023.102903}.

\bibitem[\citeproctext]{ref-baviera2016}
Baviera-Puig, Amparo, Juan Buitrago-Vera, and Carmen Escriba-Perez.
2016. {``{Geomarketing models in supermarket location strategies}.''}
\emph{Journal of Business Economics and Management} 17 (6): 1205--21.
https://doi.org/\url{http://dx.doi.org/10.3846/16111699.2015.1113198}.

\bibitem[\citeproctext]{ref-debeule2014}
De Beule, Matthias, Dirk Van den Poel, and Nico Van de Weghe. 2014.
{``An Extended Huff-Model for Robustly Benchmarking and Predicting
Retail Network Performance.''} \emph{Applied Geography} 46: 80--89.
https://doi.org/\url{https://doi.org/10.1016/j.apgeog.2013.09.026}.

\bibitem[\citeproctext]{ref-esri2025}
Esri. 2025. \emph{ArcGIS Business Analyst}. Released.
\url{https://www.esri.com/en-us/arcgis/products/arcgis-business-analyst/overview}.

\bibitem[\citeproctext]{ref-fittkau2004}
Fittkau, Dirk. 2004. {``{Beeinflussung regionaler Kaufkraftströme durch
den Autobahnlückenschluß der A 49 Kassel-Gießen - Zur empirischen
Relevanz der New Economic Geography in wirtschaftsgeographischen
Fragestellungen}.''} Dissertation, Rheinisch-Westfälische Technische
Hochschule Aachen.
https://doi.org/\url{http://dx.doi.org/10.53846/goediss-3024}.

\bibitem[\citeproctext]{ref-fuelop2011}
Fülop, Gerhard, Pascal Kopetsch, and Christian Schöpe. 2011.
{``{Catchment areas of medical practices and the role played by
geographical distance in the patient's choice of doctor}.''} \emph{The
Annals of Regional Science} 46 (3): 691--706.
https://doi.org/\url{https://doi.org/10.1007/s00168-009-0347-y}.

\bibitem[\citeproctext]{ref-haines1972}
GH, Haines Jr, Simon LS, and Alexis M. 1972. {``{Maximum Likelihood
Estimation of Central-City Food Trading Areas}.''} \emph{Journal of
Marketing Research} 9: 154--59.
https://doi.org/\url{https://doi.org/10.2307/3149948}.

\bibitem[\citeproctext]{ref-giraud2025}
Giraud, Timothée, and Hadrien Commenges. 2025. \emph{SpatialPosition:
Spatial Position Models}. Released.
\url{https://doi.org/10.32614/CRAN.package.SpatialPosition}.

\bibitem[\citeproctext]{ref-guessefeldt2002}
Güßefeldt, Jörg. 2002. {``{Zur Modellierung von räumlichen
Kaufkraftströmen in unvollkommenen Märkten}.''} \emph{ERDKUNDE} 56 (4):
351--70. \url{https://doi.org/10.3112/erdkunde.2002.04.02}.

\bibitem[\citeproctext]{ref-haklay2008}
Haklay, Mordechai. 2008. {``OpenStreetMap: User-Generated Street
Maps.''} \emph{IEEE Pervasive Computing} 7 (4): 12--18.
\url{https://doi.org/10.1109/MPRV.2008.80}.

\bibitem[\citeproctext]{ref-hansen1959}
Hansen, Walter G. 1959. {``{How Accessibility Shapes Land Use}.''}
\emph{Journal of the American Institute of Planners} 25 (2): 73--76.
\url{https://doi.org/10.1080/01944365908978307}.

\bibitem[\citeproctext]{ref-huff1964}
Huff, David L. 1961. {``{Defining and estimating a trading area}.''}
\emph{Land Economics} 28 (4): 34--38.
https://doi.org/\url{https://doi.org/10.2307/1249154}.

\bibitem[\citeproctext]{ref-huff1962}
Huff, David L. 1962. \emph{{Determination of Intra-Urban Retail Trade
Areas}}. Real Estate Research Program, Graduate Schools of Business
Administration, University of California.

\bibitem[\citeproctext]{ref-huff1963}
Huff, David L. 1963. {``{A Probabilistic Analysis of Shopping Center
Trade Areas}.''} \emph{Land Economics} 39 (1): 81--90.
https://doi.org/\url{https://doi.org/10.2307/3144521}.

\bibitem[\citeproctext]{ref-huff2003}
Huff, David L. 2003. {``{Parameter Estimation in the Huff Model}.''}
\emph{ArcUser} 6: 34--36.

\bibitem[\citeproctext]{ref-huff2008}
Huff, David L., and Bradley McCallum. 2008. \emph{{Calibrating the Huff
Model Using ArcGIS Business Analyst}}. ESRI White Paper, September 2008.
ESRI.

\bibitem[\citeproctext]{ref-jia2016}
Jia, Peng. 2016. {``Developing a Flow-Based Spatial Algorithm to
Delineate Hospital Service Areas.''} \emph{Applied Geography} 75:
137--43.
https://doi.org/\url{https://doi.org/10.1016/j.apgeog.2016.08.008}.

\bibitem[\citeproctext]{ref-latruwe2023}
Latruwe, T, M Van der Wee, P Vanleenhove, K Michielsen, S Verbrugge, and
D Colle. 2023. {``{Improving inpatient and daycare admission estimates
with gravity models}.''} \emph{Health Services and Outcomes Research
Methodology} 23 (4): 452--67.
https://doi.org/\url{https://doi.org/10.1007/s10742-022-00298-4}.

\bibitem[\citeproctext]{ref-li2012}
Li, Yingru, and Lin Liu. 2012. {``Assessing the Impact of Retail
Location on Store Performance: A Comparison of Wal-Mart and Kmart Stores
in Cincinnati.''} \emph{Applied Geography} 32 (2): 591--600.
https://doi.org/\url{https://doi.org/10.1016/j.apgeog.2011.07.006}.

\bibitem[\citeproctext]{ref-liang2020}
Liang, Yunlei, Song Gao, Yuxin Cai, Natasha Zhang Foutz, and Lei Wu.
2020. {``{Calibrating the dynamic Huff model for business analysis using
location big data}.''} \emph{Transactions in GIS} 24 (3): 681--703.
https://doi.org/\url{https://doi.org/10.1111/tgis.12624}.

\bibitem[\citeproctext]{ref-liu2022}
Liu L, Zhao Y, Lyu H. 2022. {``{An Improved Two-Step Floating Catchment
Area (2SFCA) Method for Measuring Spatial Accessibility to Elderly Care
Facilities in Xi'an, China}.''} \emph{International Journal of
Environmental Research and Public Health} 19: 11465.
https://doi.org/\url{https://doi.org/10.3390/ijerph191811465}.

\bibitem[\citeproctext]{ref-luo2003}
Luo, Wei, and Fahui Wang. 2003. {``{Measures of Spatial Accessibility to
Health Care in a GIS Environment: Synthesis and a Case Study in the
Chicago Region}.''} \emph{Environment and Planning B: Planning and
Design} 30 (6): 865--84. \url{https://doi.org/10.1068/b29120}.

\bibitem[\citeproctext]{ref-mensing2018}
Mensing, Matthias. 2018. {``{L}ebensmittel-{O}nlinehandel -
{A}lternative Zur Zukünftigen {V}ersorgung Der {B}evölkerung Ländlicher
{R}äume?''} Dissertation, Rheinisch-Westfälische Technische Hochschule
Aachen. \url{https://doi.org/10.18154/RWTH-2019-02683}.

\bibitem[\citeproctext]{ref-nakanishi1974}
Nakanishi, Masao, and Lee G. Cooper. 1974. {``{Parameter estimation for
a Multiplicative Competitive Interaction Model: Least squares
approach}.''} \emph{Journal of Marketing Research} 11 (3): 303--11.
https://doi.org/\url{https://doi.org/10.2307/3151146}.

\bibitem[\citeproctext]{ref-nakanishi1982}
Nakanishi, Masao, and Lee G. Cooper. 1982. {``{Technical Note ---
Simplified Estimation Procedures for MCI Models}.''} \emph{Marketing
Science} 1 (3): 314--22.
https://doi.org/\url{https://doi.org/10.1287/mksc.1.3.314}.

\bibitem[\citeproctext]{ref-neis2008}
Neis, Pascal, and Alexander Zipf. 2008. {``OpenRouteService.org Is Three
Times "Open": Combining OpenSource, OpenLS and OpenStreetMap.''}
\emph{GIS Research UK Conference (GISRUK 2008)}.

\bibitem[\citeproctext]{ref-orpana2003}
Orpana, Tommi, and Jouko Lampinen. 2003. {``Building Spatial Choice
Models from Aggregate Data.''} \emph{Journal of Regional Science} 43
(2): 319--48.
https://doi.org/\url{https://doi.org/10.1111/1467-9787.00301}.

\bibitem[\citeproctext]{ref-oruc2012}
Oruc, Nermin, and Boris Tihi. 2012. {``{Competitive Location Assessment
-- the MCI Approach}.''} \emph{South East European Journal of Economics
and Business} 7 (2): 35--49.
\url{https://doi.org/10.2478/v10033-012-0013-7}.

\bibitem[\citeproctext]{ref-pavlis2014}
Pavlis, Michail, Les Dolega, and Alex Singleton. 2014.
\emph{Huff-Tools}. Released.
\url{https://github.com/alexsingleton/Huff-Tools/}.

\bibitem[\citeproctext]{ref-pereira2024}
Pereira, Rafael H. M., and Daniel Herszenhut. 2024. \emph{Accessibility:
Transport Accessibility Measures}. Released.
\url{https://doi.org/10.32614/CRAN.package.accessibility}.

\bibitem[\citeproctext]{ref-rauch2023}
Rauch, S., S. Stangl, T. Haas, J. Rauh, and P. U. Heuschmann. 2023.
{``Spatial Inequalities in Preventive Breast Cancer Care: A Comparison
of Different Accessibility Approaches for Prevention Facilities in
Bavaria, Germany.''} \emph{Journal of Transport \& Health} 29: 101567.
https://doi.org/\url{https://doi.org/10.1016/j.jth.2023.101567}.

\bibitem[\citeproctext]{ref-vonrhein2025}
Rhein, M von, J Hauser, L Haldimann, R Jörg, and O. Gruebner. 2025.
{``{Imbalanced access to pediatric primary care in Switzerland:
geographic differences and modeled future challenges}.''} \emph{European
Journal of Pediatrics} 184: 648.
https://doi.org/\url{https://doi.org/10.1007/s00431-025-06441-w}.

\bibitem[\citeproctext]{ref-suarezvega2015}
Suárez-Vega, Rafael, José Luis Gutiérrez-Acuña, and Manuel
Rodríguez-Díaz. 2015. {``{Locating a supermarket using a locally
calibrated Huff model}.''} \emph{International Journal of Geographical
Information Science} 29 (2): 217--33.
\url{https://doi.org/10.1080/13658816.2014.958154}.

\bibitem[\citeproctext]{ref-subal2021}
Subal J, Krisp JM, Paal P. 2021. {``{Quantifying spatial accessibility
of general practitioners by applying a modified huff three-step floating
catchment area (MH3SFCA) method}.''} \emph{International Journal of
Health Geographics} 20: 9.
https://doi.org/\url{https://doi.org/10.1186/s12942-021-00263-3}.

\bibitem[\citeproctext]{ref-wang2022}
Wang, Huimin, Xiaojian Wei, and Weixuan Ao. 2022. {``{Assessing Park
Accessibility Based on a Dynamic Huff Two-Step Floating Catchment Area
Method and Map Service API}.''} \emph{ISPRS International Journal of
Geo-Information} 11 (7). \url{https://doi.org/10.3390/ijgi11070394}.

\bibitem[\citeproctext]{ref-wang2026}
Wang, Yao, Liushan Lin, Xiaodong Meng, Meilin Zhu, and Changcheng Kan.
2026. {``{Measuring airport catchment areas via the Huff gravity model
calibrated with mobile location data---Evidence from the Yangtze River
Delta region}.''} \emph{Journal of Transport Geography} 131: 104552.
https://doi.org/\url{https://doi.org/10.1016/j.jtrangeo.2026.104552}.

\bibitem[\citeproctext]{ref-wieland2015}
Wieland, Thomas. 2015. \emph{Räumliches Einkaufsverhalten Und
Standortpolitik Im Einzelhandel Unter Berücksichtigung von
Agglomerationseffekten - Theoretische Erklärungsansätze,
Modellanalytische Zugänge Und Eine Empirisch-Ökonometrische
Marktgebietsanalyse Anhand Eines Fallbeispiels Aus Dem Ländlichen Raum
Ostwestfalens/Südniedersachsens}. MetaGIS.

\bibitem[\citeproctext]{ref-wieland2017}
Wieland, Thomas. 2017. {``Market Area Analysis for Retail and Service
Locations with MCI.''} \emph{The R Journal} 9: 298--323.
\url{https://doi.org/10.32614/RJ-2017-020}.

\bibitem[\citeproctext]{ref-wieland2018}
Wieland, Thomas. 2018. {``{Modellgestützte Verfahren und "big (spatial)
data" in der regionalen Versorgungsforschung II: Räumliche
Interaktionsmodelle}.''} \emph{Monitor Versorgungsforschung} 11 (3):
59--64. \url{https://doi.org/10.24945/MVF.03.18.1866-0533.2083}.

\bibitem[\citeproctext]{ref-wieland2019}
Wieland, Thomas. 2019. {``{Competitive locations of grocery stores in
the local supply context - The case of the urban district
Freiburg-Haslach}.''} \emph{European Journal of Geography} 9 (3):
89--115.
\url{https://www.eurogeojournal.eu/index.php/egj/article/view/41}.

\end{CSLReferences}

\end{document}
